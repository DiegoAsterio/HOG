\documentclass[a4paper,12pt]{article}

\usepackage[spanish]{babel}
\usepackage[utf8]{inputenc}
\usepackage{enumerate}
\usepackage{amsmath}
\usepackage{extsizes}
\usepackage{amssymb}
\usepackage{dsfont}
\usepackage{graphicx}
\usepackage{cancel}
\usepackage[usenames]{color}
\usepackage[dvipsnames]{xcolor}
\usepackage{accents}
\usepackage{flushend}
\usepackage{tikz}
\usepackage[LGR,T1]{fontenc}
\newcommand{\textgreek}[1]{\begingroup\fontencoding{LGR}\selectfont#1\endgroup}
\usetikzlibrary{arrows,automata}
\usepackage{multicol}
\setlength{\columnsep}{1cm}
\usepackage{listings}
\usepackage{graphics,graphicx, float} %para incluir imágenes y colocarlas



\usepackage[hidelinks]{hyperref}

\usepackage[vmargin=3cm,hmargin=3cm]{geometry}
%\setlength\parindent{0pt}

\setlength\parindent{0pt}


% Carpeta con las imágenes
%\graphicspath{{}}

\begin{document}


	\begin{center}
		\LARGE{\textbf{Visión por Computador (2018-2019)} \\ Grado en Ingeniería Informática y Matemáticas \\ Universidad de Granada }
		\vspace*{2.5cm}

		\rule{\textwidth}{1.6pt}\vspace*{-\baselineskip}\vspace*{4pt}
		\rule{\textwidth}{1.6pt}\vspace*{-\baselineskip}\vspace*{2pt}
		\vspace{0.5cm}

		\Huge{Descriptores HOG en la detección de peatones}

		\vspace{0.5cm}
		\rule{\textwidth}{1.6pt}\vspace*{-\baselineskip}\vspace*{2pt}
		\rule{\textwidth}{1.6pt}\vspace*{-\baselineskip}\vspace*{4pt}

		\vspace{4cm}

\begin{figure}[h!]
	\centering
	\includegraphics[scale=0.75]{./Imagenes/etsiit.jpeg}
	\label{fig:logoETSIIT}
\end{figure}

		\vspace{4cm}
		\large{Ignacio Aguilera Martos y Diego Asterio de Zaballa \\ \today }

	\end{center}




\newpage

\tableofcontents

\newpage

%%%%%%%%%%%%%%%%%%%%%%%%%%%%%%%%%%%%%%%%%%%%%%%%%%%%%%%%%%%%%%%%%%%%%%%%%%%%%%%%%%%%%%
%%             Descripción del problema y el enfoque que le hemos dado              %%
%%%%%%%%%%%%%%%%%%%%%%%%%%%%%%%%%%%%%%%%%%%%%%%%%%%%%%%%%%%%%%%%%%%%%%%%%%%%%%%%%%%%%%

\section{Descripción del problema y enfoque de la resolución}

En la resolución de este trabajo nos hemos planteado el siguiente problema a resolver, dada una imagen de una persona andando por la calle (un peatón), ¿cómo podemos reconocer que es un peatón?

Esta misma cuestión fue planteada por Navneet Dalal y Bill Triggs en su paper ``Histogram of Oriented Gradients for Human Detection'' en el que explican el desarrollo de unos descriptores que aplicados a su dataset de personas obtienen unos resultados muy buenos en la detección de las mismas. 

Estos descriptores son los descriptores HOG o descriptores de histogramas basados en gradientes. Como veremos a lo largo del trabajo se han empleado distintas variaciones a la hora de hallar los descriptores por Dalal y Triggs, quedándonos nosotros con las elecciones que han resultado más fructíferas para ellos en su análisis. 

Para comenzar hay que saber que la detección de personas es un problema difícil de abordar y que actualmente resulta muy interesante en aplicaciones por ejemplo en coches, de forma que si detecta un peatón andando por delante del vehículo este entienda que tiene que detenerse. 

Las herramientas usadas en este proyecto han sido:
\begin{itemize}
	\item Python para la implementación.
	\item OpenCV y NumPy para las operaciones de los algoritmos.
	\item El módulo de SVM incluído en OpenCV para poder predecir la existencia o no de un humano en una imagen en base a los descriptores calculados.
	\item El dataset dado por los investigadores empleado en la elaboración de su artículo.
\end{itemize}

A continuación hacemos una descripción un poco más elaborada del problema propuesto.

\subsection{Problema a resolver}

\subsection{Descriptores HOG}

El problema mencionado anteriormente se ha enfrentado desde
distintos puntos de vista lo que ha dado lugar a diversos
descriptores. Nuestra implementacion se basa en el paper de
Navneet Dalal y Bill Triggs en el que desarrollan un descriptor
basado en el gradiente de una imagen llamado descriptor HOG.
El metodo se basa en calcular los gradientes de una imagen y a
continuacion construir histogramas locales basado en la
orientacion de dichos gradientes.

La idea basica sobre la que se desarrolla la tecnica es que las
caracteristicas de forma y apariencia de un objeto de forma local
se pueden resumir gracias a la distribucion de la orientacion de
los gradientes de la imagen.

A continuacion se describe la implementacion de la tecnica a
grandes rasgos. En primer lugar la ventana de la imagen sobre la
que se va a construir el vector de caracteristicas se divide en
celdas. Estas celdas son pequenas regiones de la imagen. En cada
una de estas celdas se calcula el histograma de la direccion de
los gradientes que en ella se encuentran. Esto define la re
A continuacion estas
celdas se agrupan en bloques para que sea posible normalizarlas.


diEl descriptor HOG es un vector de caracteristicas 
La construccion del descriptor HOG se apoya fundamentalmente en el gradiente de la imagen.

\subsection{Fases de la resolución}

%%%%%%%%%%%%%%%%%%%%%%%%%%%%%%%%%%%%%%%%%%%%%%%%%%%%%%%%%%%%%%%%%%%%%%%%%%%%%%%%%%%%%%
%%                           Valoración de los resultados                           %%
%%%%%%%%%%%%%%%%%%%%%%%%%%%%%%%%%%%%%%%%%%%%%%%%%%%%%%%%%%%%%%%%%%%%%%%%%%%%%%%%%%%%%%

\section{Valoración de los resultados}

%%%%%%%%%%%%%%%%%%%%%%%%%%%%%%%%%%%%%%%%%%%%%%%%%%%%%%%%%%%%%%%%%%%%%%%%%%%%%%%%%%%%%%
%%                               Propuestas de mejora                               %%
%%%%%%%%%%%%%%%%%%%%%%%%%%%%%%%%%%%%%%%%%%%%%%%%%%%%%%%%%%%%%%%%%%%%%%%%%%%%%%%%%%%%%%

\section{Trabajo futuro: propuestas de mejora}

\normalsize


% Bibliografía.
%-----------------------------------------------------------------
%\onecolumn
%\bibliography{referencias}
%\bibliographystyle{plain}
%\nocite{*}

\end{document}
